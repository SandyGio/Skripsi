%versi 2 (8-10-2016) 
\chapter{Pendahuluan}
\label{chap:intro}
   
\section{Latar Belakang}
\label{sec:label}

Bagian ini akan diisi dengan apa yang melatarbelakangi pembuatan template skripsi ini.
Termasuk juga masalah-masalah yang akan dihadapi untuk membuatnya, termasuk kurangnya kemampuan penguasaan \LaTeX{} sehingga template ini dibuat dengan mengandalkan berbagai contoh yang tersebar di dunia maya, yang digabung-gabung menjadi satu jua.
Bagian lain juga akan dilengkapi, untuk sementara diisi dengan lorem ipsum versi bahasa inggris.

\dtext{5-10}

\section{Rumusan Masalah}
\label{sec:rumusan}
Pada perangkat lunak ini, terdapat rumusan masalah sebagai berikut:
\begin{itemize}
	\item Bagaimana cara menggunakan \textit{Node.js}?
	\item Bagaimana cara mendapatkan data \textit{event} dari \textit{Outlook Calendar}?
	\item Bagaimana mengubah status pada aplikasi \textit{Slack} menggunakan \textit{Slack} API?  
	\item Bagaimana cara membuat program agar dapat mengubah status pada aplikasi \textit{Slack} di jadwal yang telah didapat dari aplikasi \textit{Outlook Calendar}? 
	
\end{itemize}

\section{Tujuan}
\label{sec:tujuan}
Adapun pada perangkat lunak ini memiliki tujuan sebagai berikut:
\begin{itemize}
	\item Mengetahui cara menggunakan \textit{Node.js}. 
	\item Mengetahui cara mendapatkan data \textit{event} dari \textit{Outlook Calendar}.   
	\item Mengetahui cara mengubah status pada aplikasi \textit{Slack} menggunakan \textit{Slack} API. 
	\item Mengetahui cara membuat program agar dapat mengubah status pada aplikasi \textit{Slack} di jadwal yang telah didapat dari aplikasi \textit{Outlook Calendar}.  
	
\end{itemize}

\section{Batasan Masalah}
\label{sec:batasan}
Untuk mempermudah pembuatan template ini, tentu ada hal-hal yang harus dibatasi, misalnya saja bahwa template ini bukan berupa style \LaTeX{} pada umumnya (dengan alasannya karena belum mampu jika diminta membuat seperti itu)

\dtext{8}

\section{Metodologi}
\label{sec:metlit}
Tentunya akan diisi dengan metodologi yang serius sehingga templatenya terkesan lebih serius.

\dtext{9}

\section{Sistematika Pembahasan}
\label{sec:sispem}
Rencananya Bab 2 akan berisi petunjuk penggunaan template dan dasar-dasar \LaTeX.
Mungkin bab 3,4,5 dapt diisi oleh ketiga jurusan, misalnya peraturan dasar skripsi atau pedoman penulisan, tentu jika berkenan.
Bab 6 akan diisi dengan kesimpulan, bahwa membuat template ini ternyata sungguh menghabiskan banyak waktu.

\dtext{10}