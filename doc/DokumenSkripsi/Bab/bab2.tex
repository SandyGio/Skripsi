%versi 2 (8-10-2016)
\chapter{Landasan Teori}
\label{chap:teori}

\section{\textit{Outlook.com Calendar API}}
\label{sec:outlookCalendar}
Untuk mengakses dan mendapatkan data yang terdapat di \textit{Outlook.com Calendar}, dibutuhkan \textit{Microsoft Graph API}. \textit{Microsoft Graph} itu sendiri adalah gerbang untuk mendapatkan data-data yang terdapat di \textit{Microsoft} 365 dan \textit{Outlook.com Calendar} termasuk di dalam layanan \textit{Office} 365. \textit{Microsoft Graph API} adalah sebuah \textit{webservice} yang memiliki satu buah \textit{endpoint} untuk memperoleh data yang ada pada \textit{Azure Active Directory}, layanan \textit{Office} 365(\textit{SharePoint, OneDrive, Outlook/Exchange, Microsoft Teams, OneNote, Planner,} dan \textit{Excel}), layanan \textit{Enterprise Mobility and Security}(\textit{Identity Manager, Intune, Advanced Threat Analytics,} dan \textit{Advanced Threat Protection}), layanan \textit{Windows} 10(\textit{activities} dan \textit{devices}), dan \textit{Education }yaitu menuju \textit{endpoint} ke \textit{https://graph.microsoft.com}. \textit{Microsoft Graph} dapat dipergunakan untuk membangun suatu aplikasi dengan memanfaatkan konteks unik dari masing-masing penggunanya. Aplikasi yang dapat dibangun dengan bantuan dari \textit{Microsoft Graph} bisa jadi memiliki fungsi seperti:

\begin{itemize}
\item Melihat pertemuan/ \textit{meeting} pengguna yang berikutnya sudah terjadwalkan serta membantu pengguna untuk mempersiapkan dengan memberikan informasi dari profil peserta. 
\item Melihat kepada kalender pengguna sehingga bisa merekomendasikan waktu yang tepat untuk pertemuan/ \textit{meeting} selanjutnya. 
\item Mengubah file \textit{Excel} yang berada di \textit{OneDrive} pengguna secara \textit{real-time} dan bisa melalui ponsel pengguna.
\item Memantau perubahan yang ada di kalender pengguna sehingga bisa memngirimkan peringatan jika pengguna menghabiskan terlalu banyak waktu untuk melakukan rapat, serta bisa juga merekomendasikan \textit{event-event} yang bisa didelegasikan atau dilewatkan berdasarkan dari seberapa relevannya \textit{event} itu kepada pengguna.
\item Membantu pengguna untuk bisa memilah informasi yang ditujukan untuk pekerjaan dengan informasi yang bersifat pribadi. Misalkan dengan pengelompokkan gambar-gambar pribadi yang masuk ke \textit{OneDrive} pribadi serta memasukkan gambar-gambar pekerjaan ke \textit{OneDrive for Business} pengguna.     
\end{itemize} 

Permintaan yang populer diminta pada \textit{Microsoft Graph API} antara lain: 
\begin{center}
\begin{tabular}{|p{2cm}|p{15cm}|}
 \hline \textbf{Operation} & \textbf{URL} \\ \hline 
 GET\newline my profile & https://graph.microsoft.com/v1.0/me \\ \hline
 GET\newline my files & https://graph.microsoft.com/v1.0/me/drive/root/children \\ \hline 
 GET\newline my photo & https://graph.microsoft.com/v1.0/me/photo/\$value \\ \hline
 GET\newline my mail & https://graph.microsoft.com/v1.0/me/messages \\ \hline 
 GET\newline my high importance email & https://graph.microsoft.com/v1.0/me/messages?\$filter=importance\%20eq\%20'high' \\ \hline 
 GET\newline my calendar events & https://graph.microsoft.com/v1.0/me/events \\ \hline 
 GET\newline my manager & https://graph.microsoft.com/v1.0/me/manager \\ \hline 
 GET\newline last user to modify file foo.txt & https://graph.microsoft.com/v1.0/me/drive/root/children/foo.txt/lastModifiedByUser \\ \hline 
 GET\newline Office365 groups I'm member of & https://graph.microsoft.com/v1.0/me/memberOf/\$/microsoft.graph.group?\$filter=\newline groupTypes/any(a:a\%20eq\%20'unified') \\ \hline 
 GET\newline users in my organization & https://graph.microsoft.com/v1.0/users \\ \hline 
 GET\newline groups in my organization & https://graph.microsoft.com/v1.0/groups \\ \hline 
 GET\newline people related to me & https://graph.microsoft.com/v1.0/me/people \\ \hline
 GET\newline items trending around me & https://graph.microsoft.com/beta/me/insights/trending \\ \hline
 GET\newline my notes & https://graph.microsoft.com/v1.0/me/onenote/notebooks \\ \hline
\end{tabular}
\end{center}

Dari \textit{Microsoft Graph} ini, barulah dapat mengakses \textit{Outlook.com Calendar API}. Cara untuk menggunakan membangun aplikasi menggunakan \textit{API} dari \textit{Microsoft} yang tersedia adalah:
\begin{description}
\item [Mendaftarkan aplikasi yang akan dibuat ke Azure AD.]\hfill \\ Untuk Mendaftarkan aplikasi yang akan dibuat, ada beberapa langkah yaitu:

\begin{enumerate}
	\item Masuk ke \textit{Microsoft App Registration Portal} dengan menggunakan akun \textit{Microsoft} atau akun sekolah atau akun kantor. 
	\item Pilih dan klik tombol bertuliskan ``\textit{Add an app}''.
	\item Masukkan nama untuk aplikasi yang akan dibuat dan klik tombol ``\textit{Create application}''.
	\item Salin ``\textit{application ID}''. \textit{Application ID} ini nantinya dipergunakan untuk mengkonfigurasikan aplikasinya. 
	\item Dibawah bagian ``\textit{Platforms}'', pilih tombol ``\textit{Add Platforms}'' dan pilihlah \textit{platform} yang sesuai untuk aplikasi. 
	
		\subitem \textbf{Untuk \textit{Native} atau \textit{Mobile apps}:}
		\begin{enumerate}
		\item Pilih ``\textit{Native Application}''.
		\item Salin bagian ``\textit{Built-in redirect URI}''. Hal ini diperlukan untuk menkonfigurasi aplikasinya. \textit{Redirect URI} ini disediakan identik untuk aplikasinya yang gunanya untuk memastikan bahwa pesan yang dikirim ke \textit{URI} hanya dikirim ke aplikasi itu. 
		\end{enumerate}	
		
		\subitem \textbf{Untuk \textit{web apps}}
		\begin{enumerate}
		\item Pilih ``\textit{Web}''.
		\item \textit{Check} kotak dengan tulisan ``\textit{Allow Implicit Flow}'' jika ingin mengaktifkan \textit{OpenID Connect Hybrid} dan \textit{implicit flow}. \textit{Implicit Flow} memungkinkan aplikasi untuk menerima \textit{sign-in info} dan juga \textit{access token}, sedangkan nilai \textit{default} dari bagian ini adalah \textit{hybrid flow} dimana \textit{flow} ini memungkinkan aplikasi untuk menerima \textit{sign-in info} yaitu \textit{token id}, dan juga \textit{artifacts} atau dalam kasus ini adalah kode otorisasi yang digunakan aplikasi untuk mendapatkan \textit{access token}.
		\item Tentukan \textit{redirect URI} yang adalah bagian dari aplikasi yang dihubungi oleh \textit{endpoint} dari \textit{Azure AD} 2.0 saat memproses permintaan otentikasi. 
		\item Dibawah ``\textit{Application Secrets}'', pilihlah ``\textit{Generate New Password}'' dan salinlah \textit{app secret} yang terdapat di \textit{New Password generated dialog box}. Perlu diketahui bahwa salinlah \textit{app secret} sebelum \textit{New password generated dialog} ditutup karena setelah ditutup, \textit{app secret} tidak dapat diambil lagi. 
		\end{enumerate}
	\item Pilihlah ``\textit{Save}''.			 
	\end{enumerate}
	
Tabel berikut menunjukkan \textit{properties} yang perlu dikonfigurasi dan disalin untuk berbagai jenis aplikasi. Nilai \textit{assigned.} berarti harus menggunakan nilai yang diberikan oleh \textit{Azure AD}. 

\item [Mendapatkan otorisasi.]\hfill \\Langkah pertama untuk mendapatkan otorisasi adalah dengan cara meminta kepada \textit{Azure AD} 2.0 \textbf{/authorize} \textit{endpoint}. Azure ID akan mengecek pengguna yang masuk dan memastikan persetujuan pengguna untuk izin permintaan aplikasi tersebut. Berikut contoh \textit{request} untuk mendapatkan \textit{authorize code}:

\begin{center}
	\begin{tabular}{|p{15cm}|}
	\hline
	https://login.microsoftonline.com/\{tenant\}/oauth2/v2.0/authorize?\newline
	client\_id=6731de76-14a6-49ae-97bc-6eba6914391e\newline
	\&response\_type=code\newline
	\&redirect\_uri=http\%3A\%2F\%2Flocalhost\%2Fmyapp\%2F\newline
	\&response\_mode=query\newline
	\&scope=offline\_access\%20user.read\%20mail.read\newline
	\&state=12345\\
	\hline
	\end{tabular}
\end{center}

\begin{center}
	\begin{tabular}{|p{2.5cm}|p{2.5cm}|p{10cm}|}
		\hline
		\textbf{Parameter}& &\textbf{Deskripsi}\\ \hline
		\textit{tenant} & \textit{required} & Nilai yang dipakai untuk mengontrol siapa yang bisa mengakses masuk ke aplikasi. Nilainya bisa diisi dengan akun \textit{Microsoft}. \\ \hline
		 \textit{client\_id} & \textit{required} & Diisi dengan \textit{Application ID} yang diberikan oleh portal registrasi (\textit{apps.dev.microsoft.com}).\\ \hline
		 \textit{response\_type} & \textit{required} & Diisi dengan nilai ``\textit{code}'' untuk menjalankan kode otorisasi.\\ \hline
		 \textit{redirect\_uri} & \textit{recommended} & Diisi dengan redirect uri yang nantinya akan menerima respon dari otentikasi yang akan diterima oleh aplikasi. Nilai disini harus sama dengan yang sudah diisi di portal registrasi kecuali \textit{URL} yang disandikan. Untuk \textit{native} dan \textit{mobile apps}, ada nilai default untuk \textit{redirect uri} yaitu menuju ke \textit{https://login.microsoftonline.com/common/oauth2/nativeclient}.\\ \hline
		 \textit{scope} & \textit{required} & Diisi dengan hak akses pada \textit{Microsoft Graph} yang akan diberikan pada pengguna aplikasi. \\ \hline
		 \textit{response\_mode} & \textit{recommended} & Nilai yang akan menentukan metode kembalian \textit{token} yang akan dihasilkan kembali ke aplikasi. Terdapat 2 nilai yaitu ``\textit{query}'' dan ``\textit{form\_post}''. \\ \hline
		 \textit{state} & \textit{recommended} & Nilai yang dipasang untuk mencegah serangan \textit{cross-site request}. Parameter ini juga bisa digunakan untuk menyandikan informasi tentang keadaan pengguna di aplikasi sebelum melakukan permintaan otentikasi.\\
		 \hline		 
	\end{tabular}
\end{center}

Setelah melakukan request untuk mendapatkan kode otentikasi, maka akan ditujukan ke halaman yang seperti ini. 

Bentuk dari respon otentikasi seperti:

\begin{center}
	\begin{tabular}{|p{15cm}|}
		\hline
		GET https://localhost/myapp/?\newline
		code=M0ab92efe-b6fd-f08-87dc-2c6500a7f84d\newline
		\&state=12345\\ \hline
	\end{tabular}
\end{center}

\begin{center}
	\begin{tabular}{|p{3cm}|p{12cm}|}
		\hline
		\textbf{Parameter}&\textbf{Deskripsi}\\ \hline
		\textit{code} & Kode otorisasi yang diminta aplikasi. Dengan kode otorisasi ini, aplikasi bisa meminta \textit{access token} untuk mendapatkan data. Kode otorisasi memiliki umur yang singkat yaitu akan \textit{expired} setelah 10 menit. \\ \hline
		 \textit{state} & Nilai yang akan bernilai sama persis seperti saat melakukan \textit{request} jika saat melakukan \textit{request} memasukkan parameter ``\textit{state}'' juga.\\
		 \hline		 
	\end{tabular}
\end{center}

Setelah mendapat \textit{authorize code}, maka otorisasi dilanjutkan dengan meminta kepada \textit{Azure AD} 2.0 \textbf{/token} \textit{endpoint} untuk mendapatkan \textit{access token} untuk aplikasi tersebut. Aplikasi yang ingin melakukan \textit{request} untuk mendapatkan \textit{access token} akan memakai kode otorisasi dan juga mengirimkan \textit{request} dengan cara ``\textit{post}''. Berikut contoh format untuk melakukan \textit{request} kepada \textit{endpoint} \textit{/token}.

\begin{center}
	\begin{tabular}{|p{15cm}|}
		\hline
		POST /common/oauth2/v2.0/token HTTP/1.1\newline
		Host: https://login.microsoftonline.com\newline
		Content-Type: application/x-www-form-urlencoded\newline
		\newline
		client\_id=6731de76-14a6-49ae-97bc-6eba6914391e\newline
		\&scope=user.read\%20mail.read\newline
		\&code=OAAABAAAAiL9Kn2Z27UubvWFPbm0gLWQJVzCTE9UkP3pSx1aXxUjq3n8b2\newline
		JRLk4OxVXr...\newline
		\&redirect\_uri=http\%3A\%2F\%2Flocalhost\%2Fmyapp\%2F\newline
		\&grant\_type=authorization\_code\newline
		\&client\_secret=JqQX2PNo9bpM0uEihUPzyrh    // NOTE: Only required for web apps\\
		\hline
	\end{tabular}
\end{center} 

\begin{center}
	\begin{tabular}{|p{3cm}|p{2cm}|p{10cm}|}
		\hline
		\textbf{Parameter}& &\textbf{Deskripsi}\\ \hline
		\textit{tenant} & \textit{required} & Nilai yang dipakai untuk mengontrol siapa yang bisa mengakses masuk ke aplikasi. Nilainya bisa diisi dengan akun \textit{Microsoft}. \\ \hline
		 \textit{client\_id} & \textit{required} & Diisi dengan \textit{Application ID} yang diberikan oleh portal registrasi (\textit{apps.dev.microsoft.com}).\\ \hline
		 \textit{grant\_type} & \textit{required} & Diisi dengan nilai ``\textit{authorization\_code}'' untuk menjalankan kode otorisasi.\\ \hline
		 \textit{scope} & \textit{required} & Diisi dengan hak akses pada \textit{Microsoft Graph} yang akan diberikan pada pengguna aplikasi. \\ \hline
		 \textit{code} & \textit{required} & Kode otorisasi yang didapat saat melakukan request pertama.\\ \hline 
		 \textit{redirect\_uri} & \textit{required} & \textit{redirect\_uri} yang sama yang digunakan untuk mendapatkan kode otorisasi. \\ \hline
		 \textit{client\_secret} & \textit{required for web apps} & \textit{App secret} yang didapat saat membuat dan mendaftarkan aplikasi di portal registrasi. \textit{Client\_secret} ini hanya berlaku untuk \textit{web apps} dan juga \textit{web APIs} karena di \textit{native app} tidak memungkinkan untuk menyimpan \textit{client\_secret} di gawai masing-masing. \\ \hline
	\end{tabular}
\end{center}

Contoh respon jika request berhasil diproses adalah seperti berikut:
\begin{center}
	\begin{tabular}{|p{15cm}|}
		\hline
		\{\newline
    	"token\_type": "Bearer",\newline
    	"scope": "user.read\%20Fmail.read",\newline
    	"expires\_in": 3600,\newline
    	"access\_token": "eyJ0eXAiOiJKV1QiLCJhbGciOiJSUzI1NiIsIng1dCI6Ik5HVEZ2ZEstZnl0\newline
    	aEV1Q...",\newline
    	"refresh\_token": "AwABAAAAvPM1KaPlrEqdFSBzjqfTGAMxZGUTdM0t4B4..."\newline
		\}\\ \hline
	\end{tabular}
\end{center} 

Setelah mendapatkan \textit{access token}, barulah akses untuk mendapatkan data-data dari \textit{Outlook.com Calendar} bisa diakses dengan menggunakan \textit{Outlook.com Calendar API}. Pentingnya langkah sebelum ini dikarenakan untuk melakukan \textit{request} kepada \textit{Outlook.com Calendar API} diperlukan parameter \textit{access token} saat hendak melakukan \textit{request}. Adapun list dari \textit{request} yang akan dipakai dari \textit{Outlook.com Calendar API} adalah:

\begin{center}
	\begin{tabular}{|p{5cm}|p{10cm}|}
	\hline
		Request & endpoint\\ \hline 
		List events & GET /me/events\\
		 & GET /users/\{id | userPrincipalName\}/events \\
		 & GET /me/calendar/events\\
		 & GET /users/\{id | userPrincipalName\}/calendar/events\\
		 & GET /me/calendars/\{id\}/events\\
		 & GET /users/\{id | userPrincipalName\}/calendars/\{id\}/events\\
		 & GET /me/calendargroup/calendars/\{id\}/events\\
		 & GET /users/\{id | userPrincipalName\}/calendargroup/calendars/\{id\}/events\\
		 & GET /me/calendargroups/\{id\}/calendars/\{id\}/events\\
		 & GET /users/\{id | userPrincipalName\}/calendargroups/\{id\}/calendars/\{id\}/events\\
	\hline
	\end{tabular}
\end{center} 

Request di atas akan meminta event yang ada di akun pengguna. Field \{id\} di atas diisi oleh id dari pengguna. Sedangkan field \{userPrincipalName\} bisa diisi dengan nama pengguna. 

\begin{lstlisting}[caption={Contoh respon dari request List events diatas},label={lst:atribute-view},language=json]

HTTP/1.1 200 OK
Content-type: application/json
Preference-Applied: outlook.timezone="Pacific Standard Time"
Content-length: 1932

{
    "@odata.context":"https://graph.microsoft.com/v1.0/$metadata#
    users('cd209b0b-3f83-4c35-82d2-d88a61820480')/events(subject,body,bodyPreview,organizer,attendees,start,end,location)",
    "value":[
        {
            "@odata.etag":"W/\"ZlnW4RIAV06KYYwlrfNZvQAAKGWwbw==\"",
            "id":"AAMkAGIAAAoZDOFAAA=",
            "subject":"Orientation ",
            "bodyPreview":"Dana, this is the time you selected for our 
            orientation. Please bring the notes I sent you.",
            "body":{
                "contentType":"html",
                "content":"<html><head></head><body><p>Dana, this is the 
                time you selected for our orientation. Please bring the 
                notes I sent you.</p></body></html>"
            },
            "start":{
                "dateTime":"2017-04-21T10:00:00.0000000",
                "timeZone":"Pacific Standard Time"
            },
            "end":{
                "dateTime":"2017-04-21T12:00:00.0000000",
                "timeZone":"Pacific Standard Time"
            },
            "location": {
                "displayName": "Assembly Hall",
                "locationType": "default",
                "uniqueId": "Assembly Hall",
                "uniqueIdType": "private"
            },
            "locations": [
                {
                    "displayName": "Assembly Hall",
                    "locationType": "default",
                    "uniqueIdType": "unknown"
                }
            ],
            "attendees":[
                {
                    "type":"required",
                    "status":{
                        "response":"none",
                        "time":"0001-01-01T00:00:00Z"
                    },
                    "emailAddress":{
                        "name":"Samantha Booth",
                        "address":"samanthab@a830edad905084922E17020313.
                        onmicrosoft.com"
                    }
                },
                {
                    "type":"required",
                    "status":{
                        "response":"none",
                        "time":"0001-01-01T00:00:00Z"
                    },
                    "emailAddress":{
                        "name":"Dana Swope",
                        "address":"danas@a830edad905084922E17020313.
                        onmicrosoft.com"
                    }
                }
            ],
            "organizer":{
                "emailAddress":{
                    "name":"Samantha Booth",
                    "address":"samanthab@a830edad905084922E17020313.
                    onmicrosoft.com"
                }
            }
        }
    ]
}
\end{lstlisting} 
	
\end{description}


\section{\textit{Slack API}}
\label{sec:slack}
Slack API adalah webservice yang akan digunakan untuk menghubungkan data yang sudah di dapat dari Outlook.com Calendar ke aplikasi Slack. Disini akan dipakai API yang berfungsi untuk mengubah status dari pengguna Slack. 


\section{\textit{Node.js}}
\label{sec:nodejs}


\section{Cron}
\label{sec:cron}
 
