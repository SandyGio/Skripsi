\chapter{Analisis}
\label{chap:analisis}

Pada bab ini akan dijelaskan mengenai analisis dari penggunaan \textit{Microsoft Graph API} dan juga penggunaan dari \textit{Slack API} serta analisis dari penggunaan \textit{cron}. 

\section{Analisis API Terkait}
\label{sec:analisis_api_terkait}

\textit{API} yang akan digunakan di skripsi ini adalah \textit{API} yang dimiliki dan disediakan oleh \textit{Microsoft Graph} dan juga \textit{API} yang dimiliki dan disediakan oleh \textit{Slack}. Maka dari itu dilakukan analisis untuk memakai \textit{API} dari kedua aplikasi terkait tersebut. 

\subsection{Analisis Microsoft Graph API}
Untuk menggunakan \textit{method-method} yang disediakan oleh \textit{Microsoft Graph API}, seperti yang sudah dijelaskan di bab sebelumnya, dibutuhkan \textit{authorization code} dan juga \textit{access token} serta aplikasi yang sudah didaftarkan terlebih dahulu yang nantinya akan diperlukan \textit{application ID} dan juga \textit{client secret} yang didapat saat mendaftarkan aplikasi. 

Setelah bisa mendapatkan \textit{authorization code} dan juga \textit{access token}, barulah setiap \textit{method} yang akan diakses bisa diminta dengan menyertakan \textit{access token} sebagai \textit{header} dari \textit{request} yang akan dijalankan. Dengan mengirimkan \textit{request} yang benar dan menyertakan \textit{access token} di \textit{header} sebagai pengidentifikasi otorisasi, maka \textit{Microsoft Graph} akan mengembalikan nilai yang seharusnya dikembalikan oleh \textit{method} yang di\textit{request} juga dengan benar. 

Analisis yang sudah dicoba dilakukan adalah dengan cara meregistrasi aplikasi dan juga melakukan sedikit perubahan terhadap \textit{template} dari contoh aplikasi \textit{web} yang sudah disediakan oleh \textit{Microsoft} yang terdapat di profil \textit{github} dari \textit{Azure AD Quick Starts}\footnote{https://github.com/AzureADQuickStarts}. \textit{Method} yang dipakai untuk meminta \textit{request} kepada aplikasi \textit{Outlook.com Calendar} untuk mendapatkan data \textit{event} yang terdapat di dalam kalender pengguna adalah \textit{method} ``\textit{List events}''. Hal ini didapatkan dari hasil pembelajaran dari dokumentasi \textit{online} yang dimiliki oleh \textit{Microsoft Graph API}, sehingga pada tahap mencoba, dilakukan uji coba mengirimkan \textit{request} dengan \textit{method} tersebut dan didapatkan hasil berupa \textit{json} yang merepresentasikan kumpulan \textit{event} yang terdapat di dalam kalender dari pengguna.

\subsection{Analisis Slack API}
Untuk dapat memakai dan mengakses \textit{method-method} yang disediakan oleh \textit{Slack}, dibutuhkan mendaftarkan aplikasi yang akan dibuat untuk bisa memperoleh \textit{Client ID} yang nantinya dibutuhkan untuk bisa mendapatkan \textit{authorization code} serta \textit{access token}. Sama seperti di \textit{Microsoft Graph API}, di dalam \textit{Slack API access token} dibutuhkan sebagai otorisasi untuk bisa mengakses \textit{method-method} yang disediakan oleh \textit{Slack}. Jika tidak memiliki \textit{access token}, maka seluruh \textit{method} yang dicoba di\textit{request} tidak akan mengembalikkan hasil dan dianggap sebagai \textit{request} yang \textit{invalid}. 

Pada sesi ini akan dibutuhkan \textit{method} dari \textit{Slack API} yang bisa mengubah status yang jika dipelajari lebih lanjut tergabung di dalam \textit{user.profile}. Dari informasi yang didapat dari membaca-baca dokumentasi \textit{online} yang disediakan oleh \textit{Slack API}, bahwa status tergabung dalam \textit{user.profile}, maka untuk memenuhi dari kebutuhan di sisi ini dicoba menggunakan \textit{method-method} yang bisa untuk mengakses kepada objek \textit{user.profile}. Karena di sisi ini akan mengubah nilai dari status, asumsi sementara dari \textit{method} yang bisa dipakai dari sisi ini adalah \textit{method users.profile.set} yang akan berguna untuk mengubah isi dari profil pengguna yang di dalamnya terdapat status. 
