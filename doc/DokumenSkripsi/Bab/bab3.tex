\chapter{Analisis}
\label{chap:analisis}

Pada bab ini akan dijelaskan mengenai analisis dari penggunaan \textit{Microsoft Graph API} untuk mendapatkan data event yang sudah tercatat di dalam Outlook Calendar web dan juga penggunaan dari \textit{Slack API} untuk memampukan program mengubah status yang terdapat pada aplikasi Slack, serta analisis dari penggunaan \textit{cron} yang akan digunakan untuk menjalankan program yang disusun secara berkala. Untuk bisa menjalankan program ini, maka dibutuhkan input dari pengguna berupa login ke Windows Live dan juga memberikan akses kepada program ini untuk mengambil access token dan refresh token yang akan digunakan untuk mengambil data event yang sudah tercatat di dalam Outlook Calendar. Selain ke Windows Live, program ini juga memerlukan pengguna untuk login ke akun yang terdapat di suatu workspace di platform Slack serta memberikan akses untuk aplikasi ini sebagai aplikasi yang terdaftar dalam workspace-nya.  

\section{Analisis Microsoft Graph API}
\label{sec:analisis_microsoft_graph_api}

Pada analisis bagian ini, dilakukan analisis mengenai API yang telah disediakan oleh Microsoft Graph API yang akan digunakan untuk mengambil data acara / event yang dibutuhkan oleh perangkat lunak yang akan dibangun. Akan ada beberapa langkah yang harus dijalankan untuk berhasil mencapai tujuan dari perangkat lunak ini. Analisis dari setiap langkah akan dijelaskan pada subbab \ref{analisis_authorization_code} sampai subbab \ref{analisis_menggunakan_refresh_token}.

\subsection{Analisis Mendapatkan Authorization Code}
\label{analisis_authorization_code}
Untuk mendapatkan authorization code, diperlukan untuk mendaftarkan aplikasi yang akan dibuat ke \textit{Microsoft App Registration Portal}\footnote{https://apps.dev.microsoft.com/}. Dari mendaftarkan aplikasi yang akan dibuat di portal registrasi tersebut akan menghasilkan Application ID, Application Secret, dan juga redirect URL yang akan digunakan. Jika platform yang dipilih adalah web, maka redirect URL harus ditentukan sendiri. Dalam meminta authorization code, aplikasi yang dibuat harus mengirimkan \textit{get request} terlebih dahulu ke \textit{endpoint /authorize} yang membutuhkan parameter seperti yang dijelaskan di tabel \ref{tab:parameter_authorization_code}. 

\begin{table}[H]
	\centering 
	\caption{Tabel parameter \textit{Authorization Code}}
	\label{tab:parameter_authorization_code}
	\begin{tabular}{|p{3cm}|p{3cm}|p{9cm}|}
	\toprule
	\textbf{Parameter} & & \textbf{Deskripsi}\\ \hline 
	\textit{tenant} & wajib & Nilai \textit{tenant} berfungsi untuk mengontrol siapa yang dapat masuk ke dalam aplikasi. Bisa diisi dengan \textit{tenant ID} atau nama domain dari akun \textit{Microsoft}.\\ \hline 
	\textit{client\_id} & wajib & Nilai yang dipakai adalah nilai dari \textit{application ID} yang didapatkan saat mendaftarkan aplikasi di \textit{Microsoft App Registration Portal}.\\ \hline 
	\textit{response\_type} & wajib & Tipe balikan yang diterima dari \textit{request}. Bernilai 				\textit{code} yang berarti akan mengembalikan \textit{code}. \\ \hline 
	\textit{redirect\_uri} & direkomendasikan & \textit{Redirect uri} dari aplikasi yang didaftarkan dimana hasil dari \textit{request} yang didapat akan dikembalikan ke url yang sudah didaftarkan. \\ \hline 
	\textit{scope} & wajib & Daftar izin dari \textit{Microsoft Graph} yang dipisahkan oleh cakupan yang diinginkan dan disetujui oleh pengguna. \\ \hline 
\textit{response\_mode} & direkomendasikan & Menentukan metode yang harus digunakan untuk mengirimkan token yang dihasilkan kembali ke aplikasi. Dapat bernilai \textit{query} atau \textit{form\_post}. \\ \hline 
	\textit{state} & direkomendasikan & Nilai yang diisi saat mengirimkan \textit{request} dan akan dikembalikan juga saat menerima \textit{response}. Tujuan dari nilai ini adalah untuk mencegah pemalsuan permintaan lintas situs. Digunakan untuk menyandikan informasi sebelum \textit{request} untuk otentikasi. Biasanya nilai ini berisi nilai unik secara acak. \\ \bottomrule
\end{tabular}  
\end{table}

Pada parameter scope, dapat diisi dengan nilai offline\_access yang akan menjadikan aplikasi mendapatkan response berupa refresh token yang berguna untuk mendapatkan access token yang baru saat yang lama sudah kadaluarsa. Contoh request yang dikirimkan akan seperti yang terdapat pada contoh table \ref{tab:contoh_request_authorization_code}.

\begin{table}[H]
	\centering 
	\caption{Tabel contoh \textit{request} \textit{Authorization Code}}
	\label{tab:contoh_request_authorization_code}
	\begin{tabular}{|p{12cm}|}
	\toprule
	https://login.microsoftonline.com/\{tenant\}/oauth2/v2.0/authorize?\\
client\_id=6731de76-14a6-49ae-97bc-6eba6914391e\\
\&response\_type=code\\
\&redirect\_uri=http\%3A\%2F\%2Flocalhost\%2Fmyapp\%2F\\
\&response\_mode=query\\
\&scope=offline\_access\%20user.read\%20mail.read\\
\&state=12345\\
	\bottomrule
\end{tabular}  
\end{table}

Dari \textit{request} seperti contoh table \ref{tab:contoh_request_authorization_code}, akan menghasilkan contoh \textit{response} seperti yang terdapat pada table \ref{tab:contoh_response_authorization_code} dengan keterangan parameter seperti yang terdapat pada table \ref{tab:parameter_response_authorization_code}. 

\begin{table}[H]
	\centering 
	\caption{Tabel contoh \textit{response} \textit{Authorization Code}}
	\label{tab:contoh_response_authorization_code}
	\begin{tabular}{|p{9cm}|}
	\toprule
	GET https://localhost/myapp/?\\
code=M0ab92efe-b6fd-df08-87dc-2c6500a7f84d\\
\&state=12345 \\
	\bottomrule
\end{tabular}  
\end{table}

\begin{table}[H]
	\centering 
	\caption{Tabel parameter \textit{response} \textit{Authorization Code}}
	\label{tab:parameter_response_authorization_code}
	\begin{tabular}{|p{3cm}|p{9cm}|}
	\toprule
	\textbf{Parameter} & \textbf{Deskripsi}\\ \hline 
	\textit{code} & Nilai ini merupakan \textit{authorization\_code} yang telah di\textit{request} oleh aplikasi. \textit{Authorization\_code} ini digunakan untuk meminta \textit{access token}. \textit{Authorization code} memiliki waktu kadaluarsa yang singkat yaitu biasanya akan kadaluarsa setelah 10 menit. \\ \hline 
	\textit{state} & Jika saat melakukan \textit{request}, parameter \textit{state} diisi, maka pada saat mengeluarkan \textit{response}, akan mengeluarkan nilai \textit{state} yang sama seperti yang sudah diisi saat melakukan \textit{request}. Aplikasi harus mengidentifikasi apakah nilai \textit{state} saat melakukan \textit{request} dengan nilai \textit{state} di \textit{response} sama atau tidak. \\ \bottomrule
\end{tabular}  
\end{table}

Hasil \textit{response} yang ditampilkan oleh table \ref{tab:contoh_response_authorization_code} muncul karena pada saat \textit{request} di table \ref{tab:contoh_request_authorization_code} terdapat parameter \textit{response\_mode} yang diisi dengan nilai \textit{query} sehingga \textit{response} yang dikembalikan dalam bentuk \textit{query string} dari \textit{redirect url}. 

\subsection{Analisis Mendapatkan Access Token}
\label{analisis_access_token}

Setelah mendapatkan authorization code, langkah selanjutnya yang harus dijalankan sebelum bisa memanggil method API yang dibutuhkan adalah dengan mendapatkan access token. Yang diperlukan untuk bisa mendapatkan access token, maka aplikasi yang dibuat membutuhkan authorization code yang diterima di langkah sebelumnya dan mengirimkan post request kepada endpoint /token. 

Untuk mengirimkan post request, diperlukan request body yang memiliki elemen-elemen seperti yang terdapat di contoh table \ref{tab:contoh_request_access_token}. Adapun penjelasan dari setiap parameter yang terdapat di dalam request body dijelaskan pada table \ref{tab:parameter_request_access_token}. 

\begin{table}[H]
	\centering 
	\caption{Tabel contoh \textit{request} \textit{Access Token}}
	\label{tab:contoh_request_access_token}
	\begin{tabular}{|p{12cm}|}
	\toprule
	POST /common/oauth2/v2.0/token HTTP/1.1\\
Host: https://login.microsoftonline.com\\
Content-Type: application/x-www-form-urlencoded\\
\\
client\_id=6731de76-14a6-49ae-97bc-6eba6914391e\\
\&scope=user.read\%20mail.read\\
\&code=OAAABAAAAiL9Kn2Z27UubvWFPbm0gLWQ\\
JVzCTE9UkP3pSx1aXxUjq3n8b2JRLk4OxVXr...\\
\&redirect\_uri=http\%3A\%2F\%2Flocalhost\%2Fmyapp\%2F\\
\&grant\_type=authorization\_code\\
\&client\_secret=JqQX2PNo9bpM0uEihUPzyrh \\ 
	\bottomrule
	\end{tabular}  
\end{table}

\begin{table}[H]
	\centering 
	\caption{Tabel parameter \textit{request} \textit{Access Token}}
	\label{tab:parameter_request_access_token}
	\begin{tabular}{|p{3cm}|p{3cm}|p{9cm}|}
	\toprule
	 \textbf{Parameter} & & \textbf{Deskripsi}\\ \hline 
	\textit{tenant} & wajib & Nilai \textit{tenant} berfungsi untuk mengontrol siapa yang dapat masuk ke dalam aplikasi. Bisa diisi dengan \textit{tenant ID} atau nama domain dari akun \textit{Microsoft}.\\ \hline 
	\textit{client\_id} & wajib & Nilai yang dipakai adalah nilai dari \textit{application ID} yang didapatkan saat mendaftarkan aplikasi di \textit{Microsoft App Registration Portal}.\\ \hline 
	\textit{grant\_type} & wajib & Harus diisi dengan nilai authorization\_code untuk alur authorization code. \\ \hline
	\textit{scope} & wajib & Daftar izin dari \textit{Microsoft Graph} yang dipisahkan oleh cakupan yang diinginkan dan disetujui oleh pengguna. Dalam langkah ini, nilai dari scope pada langkah sebelumnya harus sama dengan langkah ini.  \\ \hline 
	\textit{code} & wajib & Authorization code yang didapat dari langkah sebelumnya. \\ \hline  
	\textit{redirect\_uri} & wajib & \textit{Redirect uri} yang sama yang dipakai untuk mendapatkan authorization code. \\ \hline 
	\textit{client\_secret} & wajib untuk \textit{web apps} & Application secret yang dibuat saat mendaftarkan aplikasi di portal registrasi untuk aplikasi yang didaftarkan.\\
	\bottomrule
	\end{tabular}  
\end{table}

Dari contoh request yang dilakukan pada table\ref{tab:contoh_request_access_token}, maka akan dihasilkan contoh token seperti pada table \ref{tab:contoh_response_access_token} yang memiliki keterangan dari hasil yang dikembalikan pada table\ref{tab:parameter_response_access_token}. 

\begin{table}[H]
	\centering 
	\caption{Tabel contoh \textit{response} \textit{Access Token}}
	\label{tab:contoh_response_access_token}
	\begin{tabular}{|p{9cm}|}
	\toprule
	\{\\
    "token\_type": "Bearer",\\
    "scope": "user.read\%20Fmail.read",\\
    "expires\_in": 3600,\\
    "access\_token": "eyJ0eXAiOiJKV1QiLCJhbG\\
    ciOiJSUzI1NiIsIng1dCI6Ik5HVEZ2ZEstZnl0aEV1Q...",\\
    "refresh\_token": "AwABAAAAvPM1KaPlrEqdF\\
    SBzjqfTGAMxZGUTdM0t4B4..."\\
	\}\\ 
	\bottomrule
	\end{tabular}  
\end{table}

\begin{table}[H]
	\centering 
	\caption{Tabel parameter \textit{response} \textit{Access Token}}
	\label{tab:parameter_response_access_token}
	\begin{tabular}{|p{3cm}|p{9cm}|}
	\toprule
	 \textbf{Parameter} & \textbf{Deskripsi}\\ \hline 
	\textit{token\_type} & Menunjukkan nilai dari token. Satu-satunya jenis token yang didukung oleh Azure AD adalah Bearer.\\ \hline 
	\textit{scope} & Nilai scope yang valid untuk access\_token yang diberikan.  \\ \hline 
	\textit{expires\_in} & Lamanya access token akan berlaku(dalam detik). \\ \hline  
	\textit{access\_token} & Access token yang diminta. Dengan memakai ini, maka aplikasi bisa memanggil Microsoft Graph. \\ \hline 
	\textit{refresh\_token} & Refresh token ini berguna untuk meminta kembali access token setelah access token itu berakhir. Refresh token memiliki umur yang panjang dan dapat digunakan untuk mempertahankan akses ke source. \\
	\bottomrule
	\end{tabular}  
\end{table}

\subsection{Analisis Menggunakan Access Token untuk memanggil Microsoft Graph}
\label{analisis_menggunakan_access_token}
Setelah mendapatkan access token, panggilan ke Microsoft Graph pun bisa dilakukan dengan syarat menyertakan access token di authorization header di setiap request yang dikirim. Pada table\ref{tab:contoh_request_call_microsoft_graph} menunjukkan contoh request untuk mendapatkan profile dari pengguna yang masuk. 

\begin{table}[H]
	\centering 
	\caption{Tabel contoh \textit{request} \textit{call Microsoft Graph}}
	\label{tab:contoh_request_call_microsoft_graph}
	\begin{tabular}{|p{9cm}|}
	\toprule
	GET https://graph.microsoft.com/v1.0/me \\
Authorization: Bearer eyJ0eXAiO ... 0X2tnSQLEANnSPHY0gKcgw\\
Host: graph.microsoft.com \\
	\bottomrule
	\end{tabular}  
\end{table}

Jika request yang dikirimkan berhasil, maka akan mendapatkan response yang akan terlihat mirip dengan contoh seperti pada table \ref{tab:contoh_response_call_microsoft_graph}

\begin{table}[H]
	\centering 
	\caption{Tabel contoh \textit{response} \textit{call Microsoft Graph}}
	\label{tab:contoh_response_call_microsoft_graph}
	\begin{tabular}{|p{12cm}|}
	\toprule
	HTTP/1.1 200 OK\\
Content-Type: application/json;\\
odata.metadata=minimal;\\
odata.streaming=true;\\
IEEE754Compatible=false;\\
charset=utf-8\\
\\
request-id: f45d08c0-6901-473a-90f5-7867287de97f\\
client-request-id: f45d08c0-6901-473a-90f5-7867287de97f\\
OData-Version: 4.0\\
Duration: 727.0022\\
Date: Thu, 20 Apr 2017 05:21:18 GMT\\
Content-Length: 407\\
\\
\{\\
    "@odata.context":"https://graph.microsoft.com/v1.0/\\
    $metadata\#users/$entity",\\
    "id":"12345678-73a6-4952-a53a-e9916737ff7f",\\
    "businessPhones":[\\
        "+1 555555555"\\
    ],\\
    "displayName":"Chris Green",\\
    "givenName":"Chris",\\
    "jobTitle":"Software Engineer",\\
    "mail":null,\\
    "mobilePhone":"+1 5555555555",\\
    "officeLocation":"Seattle Office",\\
    "preferredLanguage":null,\\
    "surname":"Green",\\
    "userPrincipalName":"ChrisG@contoso.onmicrosoft.com"\\
\}\\
	\bottomrule
	\end{tabular}  
\end{table}

\subsection{Analisis Menggunakan Refresh Token untuk Mendapatkan Access Token Baru}
\label{analisis_menggunakan_refresh_token}
Access token memiliki waktu yang singkat dan ketika sudah kadaluarsa, maka aplikasi yang akan dibuat harus meminta kembali access token yang supaya bisa terus mengakses data yang ada di dalam Microsoft Graph. Cara mendapatkan access token yang baru dengan menggunakan refresh token adalah dengan cara mengirimkan post request sekali lagi kepada endpoint /token dan untuk kali ini, gunakan refresh token sebagai parameter yang dikirimkan dan juga grant type yang berisikan refresh token dalam body dari request yang dilakukan seperti contoh pada table \ref{tab:contoh_request_refresh_token} dengan keterangan parameter seperti yang dijelaskan pada table \ref{tab:parameter_request_refresh_token}. 

\begin{table}[H]
	\centering 
	\caption{Tabel contoh \textit{request} menggunakan \textit{Refresh Token}}
	\label{tab:contoh_request_refresh_token}
	\begin{tabular}{|p{12cm}|}
	\toprule
	POST /common/oauth2/v2.0/token HTTP/1.1\\
Host: https://login.microsoftonline.com\\
Content-Type: application/x-www-form-urlencoded\\
\\
client\_id=6731de76-14a6-49ae-97bc-6eba6914391e\\
\&scope=user.read\%20mail.read\\
\&refresh\_token=OAAABAAAAiL9Kn2Z27UubvWFPbm0gLWQJVzCTE\\
9UkP3pSx1aXxUjq...\\
\&redirect\_uri=http\%3A\%2F\%2Flocalhost\%2Fmyapp\%2F\\
\&grant\_type=refresh\_token\\
\&client\_secret=JqQX2PNo9bpM0uEihUPzyrh\\
	\bottomrule
	\end{tabular}  
\end{table}

\begin{table}[H]
	\centering 
	\caption{Tabel parameter \textit{request} \textit{Refresh Token}}
	\label{tab:parameter_request_refresh_token}
	\begin{tabular}{|p{3cm}|p{3cm}|p{9cm}|}
	\toprule
	 \textbf{Parameter} & & \textbf{Deskripsi}\\ \hline 
	\textit{client\_id} & wajib & Nilai yang dipakai adalah nilai dari \textit{application ID} yang didapatkan saat mendaftarkan aplikasi di \textit{Microsoft App Registration Portal}.\\ \hline 
	\textit{grant\_type} & wajib & Harus diisi dengan nilai refresh\_token. \\ \hline
	\textit{scope} & wajib & Daftar izin dari \textit{Microsoft Graph} yang dipisahkan oleh cakupan yang diinginkan dan disetujui oleh pengguna. Dalam langkah ini, nilai dari scope pada langkah meminta authorization\_code harus sama dengan langkah ini.  \\ \hline 
	\textit{refresh\_token} & wajib & Refresh token yang didapat saat merequest token yang pertama kali. \\ \hline  
	\textit{redirect\_uri} & wajib & \textit{Redirect uri} yang sama yang dipakai untuk mendapatkan authorization code. \\ \hline 
	\textit{client\_secret} & wajib untuk \textit{web apps} & Application secret yang dibuat saat mendaftarkan aplikasi di portal registrasi untuk aplikasi yang didaftarkan.\\
	\bottomrule
	\end{tabular}  
\end{table}

Jika request ini berhasil, maka akan mengembalikan response seperti pada table \ref{tab:contoh_response_refresh_token} yang memiliki keterangan parameter yang dikembalikannya seperti pada table \ref{tab:parameter_response_refresh_token}. 
\\
\begin{table}[H]
	\centering 
	\caption{Tabel contoh \textit{response} menggunakan \textit{Refresh Token}}
	\label{tab:contoh_response_refresh_token}
	\begin{tabular}{|p{12cm}|}
	\toprule
	\{\\
    "access\_token": "eyJ0eXAiOiJKV1QiLCJhbGciOiJSUzI1NiI\\
    sIng1dCI6Ik5HVEZ2ZEstZnl0aEV1Q...",\\
    "token\_type": "Bearer",\\
    "expires\_in": 3599,\\
    "scope": "user.read\%20mail.read",\\
    "refresh\_token": "AwABAAAAvPM1KaPlrEqdFSBzjq\\
    fTGAMxZGUTdM0t4B4...",\\
\}\\
	\bottomrule
	\end{tabular}  
\end{table}

\begin{table}[H]
	\centering 
	\caption{Tabel parameter \textit{response} \textit{Refresh Token}}
	\label{tab:parameter_response_refresh_token}
	\begin{tabular}{|p{3cm}|p{9cm}|}
	\toprule
	 \textbf{Parameter} & \textbf{Deskripsi}\\ \hline 
	 \textit{access\_token} & Access token yang diminta. Dengan memakai ini, maka aplikasi bisa memanggil Microsoft Graph. \\ \hline 
	\textit{token\_type} & Menunjukkan nilai dari token. Satu-satunya jenis token yang didukung oleh Azure AD adalah Bearer.\\ \hline 
	\textit{expires\_in} & Lamanya access token akan berlaku(dalam detik). \\ \hline 
	\textit{scope} & Nilai scope yang valid untuk access\_token yang diberikan.  \\ \hline  
	\textit{refresh\_token} & Refresh token ini berguna untuk meminta kembali access token setelah access token itu berakhir. Refresh token memiliki umur yang panjang dan dapat digunakan untuk mempertahankan akses ke source. \\
	\bottomrule
	\end{tabular}  
\end{table}

\subsection{Analisis Mendapatkan Data Events}
\label{sec:analisis_mendapatkan_data_events}
Data event di dalam Microsoft Graph tersimpan di dalam objek event yang memiliki relasi dengan objek user dari pengguna. Untuk dapat mengakses event yang memiliki relasi dengan user, aplikasi yang akan dibuat harus menjalankan method yang dimiliki objek user yaitu method list events mengirimkan get request kepada endpoint /me/events. List events sendiri adalah method yang berfungsi untuk mengembalikan objek-objek event yang berkaitan dengan objek user pengguna. Untuk setiap operasi get yang mengembalikan objek event di Microsoft Graph, ada sebuah parameter header ``Prefer:outlook.timezone'' yang berfungsi untuk menentukan time zone untuk mulainya dan berakhirnya event. Sebagai contoh, dapat dilihat pada table \ref{tab:contoh_header_time_zone}. 

\begin{table}[H]
	\centering 
	\caption{Tabel contoh \textit{header time zone}}
	\label{tab:contoh_header_time_zone}
	\begin{tabular}{|p{9cm}|}
	\toprule
	 Prefer: outlook.timezone="Eastern Standard Time" \\
	\bottomrule
	\end{tabular}  
\end{table}

Pada table \ref{tab:contoh_header_time_zone}, outlook timezone diatur menjadi Eastern Standard Time yang nantinya semua event yang dipanggil dengan header seperti itu akan mengembalikan starttime dan endtime dari event akan disesuaikan dengan zona waktu Eastern Standard Time. 

Untuk bisa mengakses method ini, maka diperlukan format header dari request seperti yang akan dijelaskan pada table \ref{tab:parameter_header_time_zone}. 

\begin{table}[H]
	\centering 
	\caption{Tabel parameter \textit{header time zone}}
	\label{tab:parameter_header_time_zone}
	\begin{tabular}{|p{3cm}|p{3cm}|p{9cm}|}
	\toprule
	 \textbf{Nama} & \textbf{\textit{Type}} & \textbf{Deskripsi}\\ \hline
	 \textit{Authorization} & \textit{String} & Bearer {token}. Bersifat wajib diisi. \\ \hline
	 \textit{Prefer:}& & \\
	 \textit{outlook.timezone} & \textit{String} & Digunakan untuk menentukan zona waktu yang akan dipakai untuk data yang akan dikembalikan. Bersifat optional. \\ \hline
	 \textit{Prefer:}& & \\
	 \textit{outlook.body-content-type} & \textit{String} & Merupakan nilai yang mengatur properti dari response body yang akan dikembalikan. Nilai bisa berupa ``text'' atau ``html''. Nilai default dari parameter ini adalah html. Bersifat optional. \\ \hline
	\bottomrule
	\end{tabular}  
\end{table}

Request ini juga bisa menerima parameter \$select yang berbentuk string query sebagai filter mengenai field apa saja yang mau diambil dari objek event. Dapat dilihat fungsi dari parameter string query \$select seperti pada contoh table \ref{tab:contoh_request_event} dan contoh responsenya pada table \ref{tab:contoh_response_event}. 

\begin{table}[H]
	\centering 
	\caption{Tabel contoh \textit{request event}}
	\label{tab:contoh_request_event}
	\begin{tabular}{|p{12cm}|}
	\toprule
	GET https://graph.microsoft.com/v1.0/me/events?\\
	\$select=subject,bodyPreview,organizer,start,end,location\\
	Prefer: outlook.timezone="Pacific Standard Time"\\
	\bottomrule
	\end{tabular}  
\end{table}

\begin{table}[H]
	\centering 
	\caption{Tabel contoh \textit{response event}}
	\label{tab:contoh_response_event}
	\begin{tabular}{|p{15cm}|}
	\toprule
	\begin{lstlisting}
{
    "@odata.context":"https://graph.microsoft.com/v1.0/$metadata
    #users('cd209b0b-3f83-4c35-82d2-d88a61820480')/events(subject
    ,bodyPreview,organizer,start,end,location)",
    "value":[
        {
            "@odata.etag":"W/\"ZlnW4RIAV06KYYwlrfNZvQAAKGWwbw==\"",
            "id":"AAMkAGIAAAoZDOFAAA=",
            "subject":"Orientation ",
            "bodyPreview":"Dana, this is the time you selected for 
            our orientation. Please bring the notes I sent you.",
            "start":{
                "dateTime":"2017-04-21T10:00:00.0000000",
                "timeZone":"Pacific Standard Time"
            },"end":{
                "dateTime":"2017-04-21T12:00:00.0000000",
                "timeZone":"Pacific Standard Time"
            },"location": {
                "displayName": "Assembly Hall",
                "locationType": "default",
                "uniqueId": "Assembly Hall",
                "uniqueIdType": "private"
            },"locations": [{
                    "displayName": "Assembly Hall",
                    "locationType": "default",
                    "uniqueIdType": "unknown"
            }],                
            "organizer":{
                "emailAddress":{
                    "name":"Samantha Booth",
                    "address":"samanthab@a830edad905084922E170
                    20313.onmicrosoft.com"
                }
            }
        }
    ]
}
\end{lstlisting}\\
	\bottomrule
	\end{tabular}  
\end{table}

\section{Analisis Slack API}
\label{sec:analisis_slack_api}
Untuk dapat memakai dan mengakses \textit{method-method} yang disediakan oleh \textit{Slack}, dibutuhkan mendaftarkan aplikasi yang akan dibuat untuk bisa memperoleh \textit{Client ID} yang nantinya dibutuhkan untuk bisa mendapatkan \textit{authorization code} serta \textit{access token}. Sama seperti di \textit{Microsoft Graph API}, di dalam \textit{Slack API access token} dibutuhkan sebagai otorisasi untuk bisa mengakses \textit{method-method} yang disediakan oleh \textit{Slack}. Jika tidak memiliki \textit{access token}, maka seluruh \textit{method} yang dicoba di\textit{request} tidak akan mengembalikkan hasil dan dianggap sebagai \textit{request} yang tidak \textit{valid}. 

Pada sesi ini akan dibutuhkan \textit{method} dari \textit{Slack API} yang bisa mengubah status yang terdapat pada bagian \textit{user.profile}. Dari informasi yang disediakan oleh Slack secara \textit{online}, bahwa status tergabung dalam \textit{user.profile}, maka untuk memenuhi dari kebutuhan di sisi ini dicoba menggunakan \textit{method-method} yang bisa untuk mengakses kepada objek \textit{user.profile}. Karena di sisi ini akan mengubah nilai dari status, asumsi sementara dari \textit{method} yang bisa dipakai dari sisi ini adalah \textit{method users.profile.set} yang akan berguna untuk mengubah isi dari profil pengguna yang di dalamnya terdapat status. 

\section{Analisis Cron}
\label{sec:analisis_cron}
Untuk menjalankan aplikasi yang akan dibuat untuk mengambil data dari Microsoft Graph yang berhubungan dengan pengambilan data event, maka diperlukan aplikasi yang bisa mengambil data dengan mengirimkan request kepada Microsoft Graph, tetapi pengambilan data dibutuhkan secara berkala untuk memeriksa secara berkala data yang sudah dimasukkan ke dalam Microsoft Graph. Crontab memiliki kemampuan untuk menjalankan program secara berkala sesuai dengan format pengaturan yang sudah di set dari awal pembuatan perintah crontab. Crontab tinggal harus memanggil dan menjalankan aplikasi untuk mengambil data dan secara otomatis sesuai dengan waktu yang sudah diatur di crontab, maka program akan dijalankan. Untuk contoh format file crontab akan ditunjukkan pada \ref{lst:contoh_cronfile}.

\begin{lstlisting}[caption=contoh cronfile]
\label{lst:contoh_cronfile}
# use /bin/sh to run commands, no matter what /etc/passwd says
SHELL=/bin/sh
# mail any output to 'paul', no matter whose crontab this is
MAILTO=paul
#
CRON_TZ=Japan
# run five minutes after midnight, every day
5 0 * * *       $HOME/bin/daily.job >> $HOME/tmp/out 2>&1
# run at 2:15pm on the first of every month -- output mailed to paul
15 14 1 * *     $HOME/bin/monthly
# run at 10 pm on weekdays, annoy Joe
0 22 * * 1-5    mail -s "It's 10pm" joe%Joe,%%Where are your kids?%
23 0-23/2 * * * echo "run 23 minutes after midn, 2am, 4am ..., everyday"
5 4 * * sun     echo "run at 5 after 4 every sunday"
\end{lstlisting}