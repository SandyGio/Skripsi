%versi 3 (18-12-2016)
\chapter{Kode Program}
\label{lamp:A}

%terdapat 2 cara untuk memasukkan kode program
% 1. menggunakan perintah \lstinputlisting (kode program ditempatkan di folder yang sama dengan file ini)
% 2. menggunakan environment lstlisting (kode program dituliskan di dalam file ini)
% Perhatikan contoh yang diberikan!!
%
% untuk keduanya, ada parameter yang harus diisi:
% - language: bahasa dari kode program (pilihan: Java, C, C++, PHP, Matlab, C#, HTML, R, Python, SQL, dll)
% - caption: nama file dari kode program yang akan ditampilkan di dokumen akhir
%
% Perhatian: Abaikan warning tentang textasteriskcentered!!
%
\lstinputlisting[language=Java, caption=auth.js]{./Lampiran/auth.js} 

\lstinputlisting[language=Java, caption=authorize.js]{./Lampiran/authorize.js} 

\lstinputlisting[language=Java, caption=index.js]{./Lampiran/index.js}

\lstinputlisting[language=Java, caption=slackAuthorize.js]{./Lampiran/slackAuthorize.js}

\lstinputlisting[language=Java, caption=statusChanger.js]{./Lampiran/statusChanger.js} 

\lstinputlisting[language=Java, caption=app.js]{./Lampiran/app.js} 

\lstinputlisting[language=HTML, caption=authorize\_success.hbs]{./Lampiran/authorize_success.hbs} 

\lstinputlisting[language=HTML, caption=calendar\_success.hbs]{./Lampiran/calendar_success.hbs} 

\lstinputlisting[language=HTML, caption=index.hbs]{./Lampiran/index.hbs} 

\lstinputlisting[language=HTML, caption=slack\_authorize\_success.hbs]{./Lampiran/slack_authorize_success.hbs} 


