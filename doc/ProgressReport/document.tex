\documentclass[a4paper,twoside]{article}
\usepackage[T1]{fontenc}
\usepackage[bahasa]{babel}
\usepackage{graphicx}
\usepackage{graphics}
\usepackage{float}
\usepackage[cm]{fullpage}
\pagestyle{myheadings}
\usepackage{etoolbox}
\usepackage{setspace} 
\usepackage{lipsum} 
\setlength{\headsep}{30pt}
\usepackage[inner=2cm,outer=2.5cm,top=2.5cm,bottom=2cm]{geometry} %margin
% \pagestyle{empty}

\makeatletter
\renewcommand{\@maketitle} {\begin{center} {\LARGE \textbf{ \textsc{\@title}} \par} \bigskip {\large \textbf{\textsc{\@author}} }\end{center} }
\renewcommand{\thispagestyle}[1]{}
\markright{\textbf{\textsc{Laporan Perkembangan Pengerjaan Skripsi\textemdash Sem. Genap 2015/2016}}}

\onehalfspacing
 
\begin{document}

\title{\@judultopik}
\author{\nama \textendash \@npm} 

%ISILAH DATA BERIKUT INI:
\newcommand{\nama}{Sandy Giovanni S.}
\newcommand{\@npm}{2015730041}
\newcommand{\tanggal}{01/05/2019} %Tanggal pembuatan dokumen
\newcommand{\@judultopik}{Integrasi \textit{Outlook Calender} dan \textit{Slack}} % Judul/topik anda
\newcommand{\kodetopik}{PAN4505}
\newcommand{\jumpemb}{1} % Jumlah pembimbing, 1 atau 2
\newcommand{\pembA}{Pascal Alfadian}
\newcommand{\pembB}{-}
\newcommand{\semesterPertama}{46 - Genap 18/19} % semester pertama kali topik diambil, angka 1 dimulai dari sem Ganjil 96/97
\newcommand{\lamaSkripsi}{1} % Jumlah semester untuk mengerjakan skripsi s.d. dokumen ini dibuat
\newcommand{\kulPertama}{Skripsi 1} % Kuliah dimana topik ini diambil pertama kali
\newcommand{\tipePR}{B} % tipe progress report :
% A : dokumen pendukung untuk pengambilan ke-2 di Skripsi 1
% B : dokumen untuk reviewer pada presentasi dan review Skripsi 1
% C : dokumen pendukung untuk pengambilan ke-2 di Skripsi 2

% Dokumen hasil template ini harus dicetak bolak-balik !!!!

\maketitle

\pagenumbering{arabic}

\section{Data Skripsi} %TIDAK PERLU MENGUBAH BAGIAN INI !!!
Pembimbing utama/tunggal: {\bf \pembA}\\
Pembimbing pendamping: {\bf \pembB}\\
Kode Topik : {\bf \kodetopik}\\
Topik ini sudah dikerjakan selama : {\bf \lamaSkripsi} semester\\
Pengambilan pertama kali topik ini pada : Semester {\bf \semesterPertama} \\
Pengambilan pertama kali topik ini di kuliah : {\bf \kulPertama} \\
Tipe Laporan : {\bf \tipePR} -
\ifdefstring{\tipePR}{A}{
			Dokumen pendukung untuk {\BF pengambilan ke-2 di Skripsi 1} }
		{
		\ifdefstring{\tipePR}{B} {
				Dokumen untuk reviewer pada presentasi dan {\bf review Skripsi 1}}
			{	Dokumen pendukung untuk {\bf pengambilan ke-2 di Skripsi 2}}
		}
		
\section{Latar Belakang}
Latar belakang disusunnya aplikasi ini adalah untuk membantu \textit{syncronize} akun \textit{Slack} agar bisa terlihat dalam status tertentu oleh \textit{partner}/ rekan satu tim jika \textit{user} lupa untuk mengubah status yang ada di dalam akun \textit{Slack}-nya saat terdapat jadwal di aplikasi \textit{Outlook Calendar}. 

Perangkat lunak ini akan dibuat dengan bantuan dari masing- masing API (\textit{Application Programming Interface}). \textit{Outlook Calendar} dan juga \textit{Slack} memiliki API masing- masing yang cara penggunaannya terdapat dalam dokumentasi dari aplikasi tersebut yang bisa ditemui di dalam laman \textit{website} dari masing- masing aplikasi tersebut. Perangkat ini juga akan dibangun menggunakan \textit{Node.js} yang bisa dipelajari melalui laman \textit{website} dokumentasi \textit{Node.js} itu sendiri. 

\section{Tujuan}
Tujuan dari penyusunan skripsi ini antara lain:
\begin{itemize}
	\item Mengetahui cara menggunakan \textit{Node.js}. 
	\item Mengetahui cara mendapatkan data \textit{event} dari \textit{Outlook Calendar}.   
	\item Mengetahui cara mengubah status pada aplikasi \textit{Slack} menggunakan \textit{Slack} API. 
	\item Mengetahui cara membuat program agar dapat mengubah status pada aplikasi \textit{Slack} di jadwal yang telah didapat dari aplikasi \textit{Outlook Calendar}.  
	
\end{itemize}

\section{Rumusan Masalah}
Rumusan masalah pada topik ini adalah:
\begin{itemize}
	\item Bagaimana cara menggunakan \textit{Node.js}?
	\item Bagaimana cara mendapatkan data \textit{event} dari \textit{Outlook Calendar}?
	\item Bagaimana mengubah status pada aplikasi \textit{Slack} menggunakan \textit{Slack} API?  
	\item Bagaimana cara membuat program agar dapat mengubah status pada aplikasi \textit{Slack} di jadwal yang telah didapat dari aplikasi \textit{Outlook Calendar}? 
	
\end{itemize}

\section{Detail Perkembangan Pengerjaan Skripsi}
Detail bagian pekerjaan skripsi sesuai dengan rencan kerja/laporan perkembangan terkahir :
	\begin{enumerate}
		\item \textbf{Melakukan studi literatur melalui dokumentasi \textit{online} mengenai \textit{Node.js}.}\\
		{\bf Status :} Ada sejak rencana kerja skripsi.\\
		{\bf Hasil :} Karena banyaknya library yang disediakan oleh Node.js, maka akan dilakukan studi literatur secara terus menerus seiring dengan membuat kode program untuk skripsi ini. 
		
		\item \textbf{Melakukan studi literatur melalui dokumentasi \textit{online} mengenai aplikasi \textit{Outlook Calendar}.}\\
		{\bf Status :} Ada sejak rencana kerja skripsi.\\
		{\bf Hasil :} Sudah mendapatkan bayangan dari alur garis besarnya untuk bisa memperoleh data yang dibutuhkan dari Outlook.com Calendar yaitu data tentang event. Masih akan dipelajari lebih lanjut lagi untuk kasus-kasus unik. 

		\item \textbf{Melakukan studi literatur melalui dokumentasi \textit{online} mengenai aplikasi \textit{Slack}.}\\
		{\bf Status :} Ada sejak rencana kerja skripsi.\\
		{\bf Hasil :} Secara garis besar, alur yang akan dijalankan di sisi ini sudah terbayang.

		\item \textbf{Melakukan studi literatur melalui dokumentasi \textit{online} mengenai cron dan crontab.}\\
		{\bf Status :} Baru ditambahkan pada semester ini.\\
		{\bf Hasil :} Mengetahui secara garis besar fungsi dari cron dan crontab yang nantinya akan dipakai untuk menjalankan aplikasi secara berkala.

		\item \textbf{Melakukan analisis cara melakukan \textit{synchronize} dengan aplikasi \textit{Outlook Calendar} secara berkala.}\\
		{\bf Status :} Ada sejak rencana kerja skripsi.\\
		{\bf Hasil :} Mendapatkan alur yang terurut untuk bisa meminta request dari method API yang disediakan. 

		\item \textbf{Melakukan analisis cara aplikasi merespons ketika menjadwalkan perubahan status di aplikasi \textit{Slack}, dengan kemungkinan masalah seperti : ada \textit{event} yang baru ditambahkan setelah program melakukan \textit{synchronize} secara berkala atau ada \textit{event} yang beririsan dengan \textit{event} lainnya sehingga terdapat masalah menentukan kapan status dibuang/ dikembalikan lagi ke status semula.}\\
		{\bf Status :} tidak dikerjakan.\\
		{\bf Hasil :}

		\item \textbf{Merancang bagian dari perangkat lunak yang akan mengambil data-data \textit{event} dari \textit{Outlook Calendar}. }\\
		{\bf Status :} Ada sejak rencana kerja skripsi. \\
		{\bf Hasil :} berdasarkan analisis singkat, tidak dilakukan analisis lebih jauh karena tidak diperlukan struktur data baru, karena sudah disediakan oleh OpenSteer versi terbaru

		\item \textbf{Merancang bagian dari perangkat lunak yang bertugas untuk mengubah status pada \textit{Slack} saat waktu sesuai dengan jadwal yang sudah tercatat dari \textit{Outlook Calendar}.} \\
		{\bf Status :} Ada sejak rencana kerja skripsi.\\
		{\bf Hasil :}

		\item \textbf{Mengimplementasi bagian pengambilan data dari \textit{Outlook Calendar} dan juga bagian mengatur status pada \textit{Slack} sesuai jadwal yang telah diambil kepada perangkat lunak Integrasi \textit{Outlook Calendar} dengan \textit{Slack}.}\\
		{\bf Status :} Ada sejak rencana kerja skripsi.\\
		{\bf Hasil :}

		\item \textbf{Melakukan pengujian terhadap fitur yang telah diimplementasi.}\\
		{\bf Status :} Ada sejak rencana kerja skripsi.\\
		{\bf Hasil :} 
		
		\item \textbf{Menulis dokumen skripsi}\\
		{\bf Status :} Ada sejak rencana kerja skripsi.\\
		{\bf Hasil :} \lipsum[1]
		
	\end{enumerate}

\section{Pencapaian Rencana Kerja}
Langkah-langkah kerja yang berhasil diselesaikan dalam Skripsi 1 ini adalah sebagai berikut:
\begin{enumerate}
\item Melakukan studi literatur melalui dokumentasi \textit{online} mengenai aplikasi \textit{Outlook Calendar}.
\item Melakukan studi literatur melalui dokumentasi \textit{online} mengenai aplikasi \textit{Slack}.
\item Melakukan studi literatur melalui dokumentasi \textit{online} mengenai cron dan crontab. 
\end{enumerate}



\section{Kendala yang Dihadapi}
%TULISKAN BAGIAN INI JIKA DOKUMEN ANDA TIPE A ATAU C
Kendala - kendala yang dihadapi selama mengerjakan skripsi :
\begin{itemize}
	\item Sulitnya fokus untuk mengerjakan. 
	\item Terlalu banyak godaan berupa hiburan (game, film, dll). 
	\item Dokumentasi dari Outlook Calendar kurang jelas dan ada beberapa versi dokumentasi mengacu kepada versi API yang disediakan.
\end{itemize}

\vspace{1cm}
\centering Bandung, \tanggal\\
\vspace{2cm} \nama \\ 
\vspace{1cm}

Menyetujui, \\
\ifdefstring{\jumpemb}{2}{
\vspace{1.5cm}
\begin{centering} Menyetujui,\\ \end{centering} \vspace{0.75cm}
\begin{minipage}[b]{0.45\linewidth}
% \centering Bandung, \makebox[0.5cm]{\hrulefill}/\makebox[0.5cm]{\hrulefill}/2013 \\
\vspace{2cm} Nama: \pembA \\ Pembimbing Utama
\end{minipage} \hspace{0.5cm}
\begin{minipage}[b]{0.45\linewidth}
% \centering Bandung, \makebox[0.5cm]{\hrulefill}/\makebox[0.5cm]{\hrulefill}/2013\\
\vspace{2cm} Nama: \pembB \\ Pembimbing Pendamping
\end{minipage}
\vspace{0.5cm}
}{
% \centering Bandung, \makebox[0.5cm]{\hrulefill}/\makebox[0.5cm]{\hrulefill}/2013\\
\vspace{2cm} Nama: \pembA \\ Pembimbing Tunggal
}
\end{document}

